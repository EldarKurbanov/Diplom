
{\actuality}

{\progress}
% Этот раздел должен быть отдельным структурным элементом по
% ГОСТ, но он, как правило, включается в описание актуальности
% темы. Нужен он отдельным структурынм элемементом или нет ---
% смотрите другие диссертации вашего совета, скорее всего не нужен.

{\aim} данной работы является создание системы автоматического управления робота с учётом данных, получаемых от окружающего пространства и прежде всего создание самого тестируемого образца робота и его аппаратной системы управления. 

Для~достижения поставленной цели необходимо было решить следующие {\tasks}:
\begin{enumerate}
  \item Исследовать предметную область робототехники\footnote{Робототехника не изучалась на протяжении всего курса обучения в университете.} (аппаратную и программную часть);
  \item Изучить существующие известные аналоги (в т.ч. зарубежные) и продумать как сделать робота ещё лучше;
  \item Закупить необходимое оборудование, уложившись при этом в маленький бюджет; 
  \item Разработать схему управления роботом и соответствующее ПО;
  \item Протестировать созданное изделие.
\end{enumerate}


{\novelty}
\begin{enumerate}
  \item Впервые в России был сделан робот с одновременным использованием технологии YDLIDAR, движением и распознаванием объектов окружающего пространства на базе платформы NVIDIA Jetson NANO\footnote{Возможно, это происходит не впервые, но других таких известных случаев не нашлось};
  \item Создана программно-аппаратная база, на основе которой можно сделать робота, выполняющего иной функционал.
\end{enumerate}

{\influence} данной работы заключается в том, что была решена задача создания своего собственного алгоритма движения для робота на базе относительно новой и ещё мало изученной платформы Jetson NANO со своим алгоритмом езды и следованием за целевыми объектами.

{\methods} При разработке данной системы управления и формирования поведенческой стратегии автономного мобильного робота использовались такие методы эмпирического исследования, как наблюдение и эксперимент, а к методам теоретического исследования - анализ и синтез и восхождение от абстрактного к конкретному.
