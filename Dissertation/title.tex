% Команда для линии с нижним подчёркиванием, возможностью написания сверху текста на ней и текстом под ней.
\newcommand\superunderline[3]{$\underset{\text{#3}}{\text{\underline{\hspace{0.3cm}#1\hspace{#2}}}}$}

% Тоже самое, но ещё с отступом слева от текста (выравнивание по середине)
\newcommand\superunderlinec[3]{$\underset{\text{#3}}{\text{\underline{\hspace{#2}#1\hspace{#2}}}}$}

% "Прокаченные" типы колонок для LaTeX. Умеют выравнивать содержимое столбца по горизонтали и принимать фиксированный размер. Полезная вещь.
\newcolumntype{A}[1]{>{\raggedright\let\newline\\\arraybackslash\hspace{0pt}}m{#1}} % Выравнивание по левому краю.
\newcolumntype{B}[1]{>{\centering\let\newline\\\arraybackslash\hspace{0pt}}m{#1}} % Выравнивание по середине.
\newcolumntype{D}[1]{>{\raggedleft\let\newline\\\arraybackslash\hspace{0pt}}m{#1}} % Выравнивание по правому краю.

% Титульный лист (ГОСТ Р 7.0.11-2001, 5.1)
\thispagestyle{empty}
\begin{center}
\thesisOrganization
\end{center}
%
\vspace{0pt plus4fill} %число перед fill = кратность относительно некоторого расстояния fill, кусками которого заполнены пустые места
\hfill\begin{minipage}{0.4\textwidth}
			Допустить работу к защите \\
			Зав. каф. КНЭМ \\
			\underline{\hspace{3.3cm}}Клячин В.А.\\ [2mm]
			«\underline{\hspace{0.7cm}}» \underline{\hspace{2.5cm}} 2020 г.
		\end{minipage}
%
\vspace{0pt plus4fill} %число перед fill = кратность относительно некоторого расстояния fill, кусками которого заполнены пустые места
\begin{center}
{\large \thesisAuthor}
\end{center}
%
\vspace{0cm plus0fill} %число перед fill = кратность относительно некоторого расстояния fill, кусками которого заполнены пустые места
\begin{center}
\textbf {\large %\MakeUppercase
\thesisTitle}

\vspace{0pt plus2fill} %число перед fill = кратность относительно некоторого расстояния fill, кусками которого заполнены пустые места
{%\small
Выпускная квалификационная работа бакалавра \\ по направлению \thesisSpecialtyNumber\ \\ <<\thesisSpecialtyTitle>>
}

\ifdefined\thesisSpecialtyTwoNumber
{%\small
Специальность \thesisSpecialtyTwoNumber\ "---

<<\thesisSpecialtyTwoTitle>>
}
\fi

\vspace{0pt plus2fill} %число перед fill = кратность относительно некоторого расстояния fill, кусками которого заполнены пустые места
\begin{tabular}{A{4.8cm}A{4.8cm}A{4.8cm}}
%\begin{tabular}{m{4.8cm}m{4.8cm}m{4.8cm}}
	Студент & \thesisAuthorLastName \  \thesisAuthorInitials & \superunderlinec{\phantom{28.06.2020, Курбанов}}{0.3cm}{(дата, подпись)} \\
	Научный руководитель & д.ф. - м.н., проф. каф. КНЭМ Клячин В. А. & \superunderlinec{\phantom{28.06.2020, Курбанов}}{0.3cm}{(дата, подпись)} \\
	Нормоконтролер & к. ф.-м. н., доц. каф. КНЭМ Полубоярова Н.М. & \superunderlinec{\phantom{28.06.2020, Курбанов}}{0.3cm}{(дата, подпись)} \\
	Рецензент & к.т.н.,  ст. преп. каф. Радиофизики Глухов А.Ю. & \superunderlinec{\phantom{28.06.2020, Курбанов}}{0.3cm}{(дата, подпись)}
\end{tabular}
\end{center}
%
\vspace{0pt plus4fill} %число перед fill = кратность относительно некоторого расстояния fill, кусками которого заполнены пустые места

\vspace{0pt plus4fill} %число перед fill = кратность относительно некоторого расстояния fill, кусками которого заполнены пустые места
{\centering\textbf{\thesisCity}\  \textbf{\thesisYear}\par}
