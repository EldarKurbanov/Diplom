% Команда для линии с нижним подчёркиванием, возможностью написания сверху текста на ней и текстом под ней.
\newcommand\superunderline[3]{$\underset{\text{#3}}{\text{\underline{\hspace{0.3cm}#1\hspace{#2}}}}$}

% Тоже самое, но ещё с отступом слева от текста (выравнивание по середине)
\newcommand\superunderlinec[3]{$\underset{\text{#3}}{\text{\underline{\hspace{#2}#1\hspace{#2}}}}$}

% "Прокаченные" типы колонок для LaTeX. Умеют выравнивать содержимое столбца по горизонтали и принимать фиксированный размер. Полезная вещь.
\newcolumntype{A}[1]{>{\raggedright\let\newline\\\arraybackslash\hspace{0pt}}m{#1}} % Выравнивание по левому краю.
\newcolumntype{B}[1]{>{\centering\let\newline\\\arraybackslash\hspace{0pt}}m{#1}} % Выравнивание по середине.
\newcolumntype{D}[1]{>{\raggedleft\let\newline\\\arraybackslash\hspace{0pt}}m{#1}} % Выравнивание по правому краю.

% Титульный лист (ГОСТ Р 7.0.11-2001, 5.1)
\thispagestyle{empty}
\begin{center}
\thesisOrganization
\end{center}
%
%\vspace{0pt plus1fill} %число перед fill = кратность относительно некоторого расстояния fill, кусками которого заполнены пустые места


%\vspace{0pt plus4fill} %число перед fill = кратность относительно некоторого расстояния fill, кусками которого заполнены пустые места
%\begin{center}
%{\large \thesisAuthor}
%\end{center}
%
\begin{center}
\textbf { \MakeUppercase
\thesisTitle}

%\vspace{0pt plus1fill} %число перед fill = кратность относительно некоторого расстояния fill, кусками которого заполнены пустые места
{%\small
ВЫПУСКНАЯ КВАЛИФИКАЦИОННАЯ РАБОТА \\ (бакалаврская работа) \\ по направлению подготовки  \thesisSpecialtyNumber\ \thesisSpecialtyTitle \\ профиль <<Параллельное программирование>>
}

\ifdefined\thesisSpecialtyTwoNumber
{%\small
Специальность \thesisSpecialtyTwoNumber\ "---

<<\thesisSpecialtyTwoTitle>>
}
\fi

\vspace{0pt plus4fill} %число перед fill = кратность относительно некоторого расстояния fill, кусками которого заполнены пустые места

\hfill\begin{minipage}{0.55\textwidth}
			\textbf{ВЫПОЛНИЛ:} \\
			студент гр. МОС-161 \\
			\thesisAuthorLastName\ \thesisAuthorOtherNames \\
			\underline{\hspace{\textwidth}}
		\end{minipage}
%

\vspace{0pt plus1fill} %число перед fill = кратность относительно некоторого расстояния fill, кусками которого заполнены пустые места

\hfill\begin{minipage}{0.55\textwidth}
			\textbf{НАУЧНЫЙ РУКОВОДИТЕЛЬ:} \\
			зав. кафедрой \textit{КНЭМ} \\
			д.ф.-м.н., профессор \\
			Клячин Владимир Александрович \\
			\underline{\hspace{\textwidth}}
		\end{minipage}
%

\vspace{0pt plus1fill} %число перед fill = кратность относительно некоторого расстояния fill, кусками которого заполнены пустые места

\hfill\begin{minipage}{0.55\textwidth}
			\textbf{КОНСУЛЬТАНТ:} \\
			ст. преп. кафедры \textit{КНЭМ} \\
			Гордеев Алексей Юрьевич \\
			\underline{\hspace{\textwidth}}
		\end{minipage}
%

\vspace{0pt plus1fill} %число перед fill = кратность относительно некоторого расстояния fill, кусками которого заполнены пустые места

\hfill\begin{minipage}{0.55\textwidth}
			\textbf{РАБОТА ДОПУЩЕНА К ЗАЩИТЕ:} \\
			зав. кафедрой \textit{КНЭМ} \\
			д.ф.-м.н., профессор \\
			Клячин Владимир Александрович \\
			\underline{\hspace{\textwidth}}
		\end{minipage}
%

\hfill\begin{minipage}{0.55\textwidth}
			<<30>> мая 2020 г. \\
			(протокол № \underline{\hspace{1cm}} заседания кафедры)
		\end{minipage}
%

\vspace{0pt plus4fill} %число перед fill = кратность относительно некоторого расстояния fill, кусками которого заполнены пустые места
\end{center}
%
{\centering\textbf{\thesisCity}\  \textbf{\thesisYear}\par}
