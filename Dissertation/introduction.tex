\chapter*{Введение}                         % Заголовок
\addcontentsline{toc}{chapter}{Введение}    % Добавляем его в оглавление

\newcommand{\actuality}{Актуальность данной работы обусловлена общей автоматизацией и <<роботизацией>> деятельности человека в условиях современной реальности \cite{сергеев2018стратегия}. Решение поставленной задачи позволит в дальнейшем создать робота, умеющего не только объезжать разного вида помещения, но и ещё выполнять какую-либо полезную функцию. Например, распознавание опасных объектов в окружении или исследование состава атмосферы в каком-либо замкнутом пространстве.}
\newcommand{\progress}{В настоящий момент поставленная данной работой задача формирования поведенческой стратегии и управления роботом выполнена полностью. Однако, она требует значительных улучшений для каких-либо конкретных условий работы. Например, если испытуемый робот окажется на улице, то может случиться так, что целевой объект может быть так и не найден, в связи с тем, что окружающее пространство окажется слишком широким для угла обзора камеры, установленной на робота. Соответственно, данный конкретный случай должен быть учтён в алгоритме движения робота, но это не является целью данной работы.}
\newcommand{\aim}{{\textbf\aimTXT}}
\newcommand{\tasks}{\textbf{\tasksTXT}}
\newcommand{\novelty}{\textbf{\noveltyTXT}}
\newcommand{\influence}{\textbf{\influenceTXT}}
\newcommand{\methods}{\textbf{\methodsTXT}}
\newcommand{\defpositions}{\textbf{\defpositionsTXT}}
\newcommand{\reliability}{\textbf{\reliabilityTXT}}
\newcommand{\probation}{\textbf{\probationTXT}}
\newcommand{\contribution}{\textbf{\contributionTXT}}
\newcommand{\publications}{\textbf{\publicationsTXT}}


{\actuality} 

\fixme{Здесь, короче, введение... }

{\progress}
% Этот раздел должен быть отдельным структурным элементом по
% ГОСТ, но он, как правило, включается в описание актуальности
% темы. Нужен он отдельным структурынм элемементом или нет ---
% смотрите другие диссертации вашего совета, скорее всего не нужен.

{\aim} данной работы является создание системы автоматического управления робота с учётом данных, получаемых от окружающего пространства, а также создание самого тестируемого образца робота и его аппаратной системы управления. 

Для~достижения поставленной цели необходимо было решить следующие {\tasks}:
\begin{enumerate}
  \item Исследовать предметную область робототехники\footnote{Робототехника не изучалась на протяжении всего курса обучения в университете.} (аппаратную и программную часть);
  \item Изучить существующие известные аналоги (в т.ч. зарубежные) и продумать как сделать робота ещё лучше;
  \item Закупить необходимое оборудование, уложившись при этом в маленький бюджет; 
  \item Разработать схему управления роботом и соответствующее ПО;
  \item Протестировать созданное изделие.
\end{enumerate}


{\novelty}
\begin{enumerate}
  \item Впервые был сделан робот с одновременным использованием технологий сканирования местности, движением и распознаванием объектов окружающего пространства на базе платформы NVIDIA Jetson NANO\footnote{Возможно, это происходит не в первые, но других таких известных случаев не нашлось};
  \item Создана программно-аппаратная база, на основе которой можно сделать робота, выполняющего иной функционал.
\end{enumerate}

{\influence} данной работы заключается в том, что была решена задача создания своего собственного робота на базе относительно новой и ещё мало изученной платформы Jetson NANO со своим алгоритмом езды и следованием за целевыми объектами. 

{\methods} При разработке данной системы управления и формирования поведенческой стратегии автономного мобильного робота использовались такие методы эмпирического исследования, как наблюдение и эксперимент, а к методам теоретического исследования - анализ и синтез и восхождение от абстрактного к конкретному.
 % Характеристика работы по структуре во введении и в автореферате не отличается (ГОСТ Р 7.0.11, пункты 5.3.1 и 9.2.1), потому её загружаем из одного и того же внешнего файла, предварительно задав форму выделения некоторым параметрам

\textbf{Объем и структура работы.} Выпускная квалификационная работа состоит из~введения, трёх глав,
заключения и~двух приложений.
%% на случай ошибок оставляю исходный кусок на месте, закомментированным
%Полный объём диссертации составляет  \ref*{TotPages}~страницу
%с~\totalfigures{}~рисунками и~\totaltables{}~таблицами. Список литературы
%содержит \total{citenum}~наименований.
%
Полный объём ВКР составляет
\formbytotal{TotPages}{страниц}{у}{ы}{}, включая
\formbytotal{totalcount@figure}{рисун}{ок}{ка}{ков}. Список литературы содержит
\formbytotal{citenum}{наименован}{ие}{ия}{ий}.
