\thispagestyle{empty}
\begin{center}
	МИНИСТЕРСТВО ОБРАЗОВАНИЯ И НАУКИ РФ
	
	ФЕДЕРАЛЬНОЕ ГОСУДАРСТВЕННОЕ АВТОНОМНОЕ 
	
	ОБРАЗОВАТЕЛЬНОЕ УЧРЕЖДЕНИЕ ВЫСШЕГО ОБРАЗОВАНИЯ
	
	«ВОЛГОГРАДСКИЙ ГОСУДАРСТВЕННЫЙ УНИВЕРСИТЕТ»
	
	КАФЕДРА КОМПЬЮТЕРНЫХ НАУК И ЭКСПЕРИМЕНТАЛЬНОЙ МАТЕМАТИКИ	
\end{center}

\vspace{0.5cm}

\hfill\begin{minipage}{0.51\textwidth}
	д.ф.-м.н., профессор \\
	\textbf{Клячин Владимир Александрович} \\
	<<\underline{\hspace{1cm}}>>\underline{\hspace{5cm}}2020г.
\end{minipage}

\vspace{1cm}

\begin{center}
ЗАДАНИЕ

На выполнение выпускной квалификационной работы бакалавра

Студента Курбанова Эльдара Ровшановича группы МОС-161
\end{center}

\vspace{0.3cm}

\noindent Тема: Система управления и формирования поведенческой стратегии автономного мобильного
робота на основе визуального анализа окружающего пространства.

\noindent Цель: Создание системы автоматического управления роботом на основе визуального анализа окружающего пространства, а также создание самого тестируемого образца робота.

\noindent Основные задачи:

\begin{enumerate}
	\item Исследовать предметную область робототехники;
	\item Изучить существующие известные аналоги и придумать как сделать робота лучше;
	\item Разработать схему управления роботом и соответствующее ПО;
	\item Протестировать созданное изделие.
\end{enumerate}	

\noindent Основные этапы:

\begin{enumerate}
	\item Изучение предметной области робототехники;
	\item Подбор комплектующих;
	\item Определение поведенческой стратегии будущего робота;
	\item Изучение фреймворка Robot Operating System;
	\item Создание экземпляра робота, на котором будет проводится разработка;
	\item Установка и настройка ПО;
	\item Создание схемы управления роботом, а также сервиса GPIO;
	\item Создание вспомогательных узлов, а также узла движения робота;
	\item Тестирование созданного изделия.
\end{enumerate}

\vspace{0.3cm}

\noindent Рекомендуемая литература:

\begin{enumerate}
	\item Joseph, L. Mastering ROS for Robotics Programming: Design, build, and simulate complex robots using the Robot Operating System, 2nd Edition [Текст] / L. Joseph, J. Cacace. — Packt Publishing, 2018. — 580 с. — Текст: непосредственный;
	
	\item Электронный ресурс: официальная документация ROS : сайт / Willow Garage, Inc. — URL: \url{http://wiki.ros.org/Documentation}. - Текст : электронный;
	
	\item Real-Time Loop Closure in 2D LIDAR SLAM / W. Hess [и др.] // 2016 IEEE International Conference on Robotics and Automation (ICRA). — Текст: электронный. — 2016. — С. 1271—1278. — URL: \url{https://storage. googleapis.com/pub-tools-public-publication-data/pdf/45466.pdf} (дата обр. 20.05.2020).
\end{enumerate}